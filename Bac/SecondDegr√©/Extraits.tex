\documentclass[10pt]{article}
\usepackage[T1]{fontenc}
\usepackage[utf8]{inputenc}
\usepackage{fourier}
\usepackage[scaled=0.875]{helvet} 
\renewcommand{\ttdefault}{lmtt} 
\usepackage{amsmath,amssymb,makeidx}
\usepackage[normalem]{ulem}
\usepackage{fancybox,graphicx}
\usepackage{tabularx}
\usepackage{ulem}
\usepackage{dcolumn}
\usepackage{textcomp}
\usepackage{diagbox}
\usepackage{tabularx}
\usepackage{lscape}
\newcommand{\euro}{\eurologo{}}
%Tapuscrit : Denis Vergès relu et corrigé par François Hache
\usepackage{pstricks,pst-plot,pst-text,pst-tree,pstricks-add}
\setlength\paperheight{297mm}
\setlength\paperwidth{210mm}
\setlength{\textheight}{25cm}
\newcommand{\R}{\mathbb{R}}
\newcommand{\N}{\mathbb{N}}
\newcommand{\D}{\mathbb{D}}
\newcommand{\Z}{\mathbb{Z}}
\newcommand{\Q}{\mathbb{Q}}
\newcommand{\C}{\mathbb{C}}

\renewcommand{\theenumi}{\textbf{\arabic{enumi}}}
\renewcommand{\labelenumi}{\textbf{\theenumi.}}
\renewcommand{\theenumii}{\textbf{\alph{enumii}}}
\renewcommand{\labelenumii}{\textbf{\theenumii.}}

\newcommand{\vect}[1]{\mathchoice%
{\overrightarrow{\displaystyle\mathstrut#1\,\,}}%
{\overrightarrow{\textstyle\mathstrut#1\,\,}}%
{\overrightarrow{\scriptstyle\mathstrut#1\,\,}}%
{\overrightarrow{\scriptscriptstyle\mathstrut#1\,\,}}}
\def\Oij{$\left(\text{O},~\vect{\imath},~\vect{\jmath}\right)$}
\def\Oijk{$\left(\text{O},~\vect{\imath},~\vect{\jmath},~\vect{k}\right)$}
\def\Ouv{$\left(\text{O}\,;~\vect{u},~\vect{v}\right)$}
\setlength{\voffset}{-1,5cm}
\usepackage{fancyhdr}
\usepackage{hyperref}
\hypersetup{%
pdfauthor = {APMEP},
pdfsubject = {Baccalauréat S},
pdftitle = {Métropole--La Réunion 22 juin 2015},
allbordercolors = white,
pdfstartview=FitH}   
\thispagestyle{empty}
\usepackage[frenchb]{babel}
\usepackage[np]{numprint}
\begin{document}
\setlength\parindent{0mm}
\marginpar{\rotatebox{90}{\textbf{A. P{}. M. E. P{}.}}}
\rhead{\textbf{A. P{}. M. E. P{}.}}
\lhead{\small Baccalauréat S }
\rfoot{\small{Métropole--La Réunion}}
\lfoot{\small{22 juin 2015}}
\renewcommand \footrulewidth{.2pt}
\pagestyle{fancy}
\thispagestyle{empty}
\begin{center} {\Large{\textbf{\decofourleft~Baccalauréat S Métropole--La Réunion 22 juin 2015~\decofourright
}}} 

\end{center}

\vspace{0,5cm}

\vspace{0,5cm}

\textbf{\textsc{Exercice 3 \hfill 5 points}}

\textbf{Candidats n'ayant pas suivi l'enseignement de spécialité} 

\medskip

\begin{enumerate}
 \item  Résoudre dans l'ensemble $\C$ des nombres complexes l'équation (E) d'inconnue $z$ :
\[z^2 - 8z + 64 = 0.\]
 
\end{enumerate}


\end{document}